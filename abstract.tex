% EDITING GUIDELINES
%
% * Limit lines to no more than 100 characters.  The default should be
%   80 characters.  This leaves room for the addition of 20 new
%   characters on the same line later on.
%
%   RATIONALE: Long lines are hard to read and cause trouble when
%   viewing two versions of the file side by side (e.g., using vimdiff)
%
% * Start every sentence on a new line.
%
%   RATIONALE: This makes sentences so much easier to find using forward
%   search in emacs or vim.
%
% * Do not justify paragraphs.  Particularly, when editing an existing
%   version of the text, do not change line breaks apart from adding
%   line breaks for lines that would otherwise be too long.
%
%   RATIONALE: This makes changes from previous versions so much easier
%   to identify, particularly when using a VCS.  Adding a word and
%   justifying the paragraph may change all the lines in the paragraph,
%   making this look like a huge edit.  Without justification, the
%   diff shows only the addition of the one word.

\begin{abstract}\noindent
  We present a technique for representing bounded-degree planar graphs
  in a succinct fashion while permitting I/O-efficient traversal of
  paths.
  Using our representation, a graph with $N$ vertices, each
  with an associated key of $\bitsPerKey = \OhOf{\lg N}$ bits,\footnote{In this
    paper $\lg{N}$ denotes $\log_2{N}$.} can be stored in $N\bitsPerKey
  + \OhOf{N} + \ohOf{N\bitsPerKey}$ bits and traversing a path of length $K$ 
  takes $\OhOf{K
  / \lg B}$ I/Os, where $B$ denotes the disk block size.
  By applying our construction to the dual of a terrain represented as a
  triangular irregular network, we can represent the terrain in the
  above space bounds and support path traversals on the terrain using
  $\OhOf{K / \lg B}$ I/Os, where $K$ is the number of triangles visited by
  the path.
  This is useful for answering a number of queries on the
  terrain, such as reporting terrain profiles, trickle paths, and
  connected components.
\end{abstract}

%%% Local Variables:
%%% TeX-PDF-mode: t
%%% TeX-master: "main"
%%% End:
