% --------------------------------------------------------------------------------
\newpage
\section{Appendix}\label{sec_appendix}
% --------------------------------------------------------------------------------

The following tables provide a summary of the various notations used in this paper,
in particular in Section~\ref{sec:graph_rep}.

\begin{table*}[ht]
	\centering
		\begin{tabular}{ l | l}
			Notation & Description \\ \hline
			$\perm[i,j]$ & Order of vertices within a sub-region. \\
			$B$ & Block size in number of elements. \\
			$\succblksize$ & Block in number of elements for a succinctly encoded block. \\
			$G$ & A planar graph of bounded degree on N vertices. \\
			$\glbl{x}$ & Unique graph label of vertex $x$.\\
			$\reglbl[i]{x}$ & Unique region label of vertex $x$ in region $R_i$. \\
			$\subreglbl[i,j]{x}$ & Unique region label of vertex $x$ in sub-region $R_{i,j}$. \\
			& $\alpha$-neighbourhood. \\	
			$N_\alpha(x)$ & The $\alpha$-neighbourhood of vertex $x$. \\
			$q$ & Key size in bits. \\
			$q_i$ & The number of sub-regions in region $R_i$ \\
			$t$ & The number of regions in $G$. \\
			$\numv[i]$ & Denotes the number of vertices in region $\reg[i]$. \\
			$\dnumv[i]$ & Denotes the total number of vertices in all sub-regions of~$\reg[i]$.\\
			  & (This differs from $\numv[i]$ in that duplicates maybe counted several times)\\
			$\dnumv$ & Denote the total number of vertices in all subregions of $G$.\\
			$\reg[i]$ & Region $i$, a subgraph of $G$ in the top level parition. \\
			$\subreg[i,j]$ & Sub region $i,j$, a subgraph of $G$ and subset of region $R_i$. \\
			$\numv[i,j]$ & Denotes the number of vertices in sub-region $\subreg[i,j]$. \\
			$s_R$ & Upper bound on the number of bits required to encode a sub-region \\
			& $\alpha$-neighbourhood. \\
			$s_S$ & Upper bound on the number of bits required to encode a region \\		
			$S^{R}_\alpha$ & Set of region boundary vertices selected \\
			& to build $\alpha$-neigbhourhoods for (Section~\ref{sec:alt_block_scheme}). \\
			$S^{SR}_\alpha$ & Set of sub-region boundary vertices selected \\
			& to build $\alpha$-neigbhourhoods for (Section~\ref{sec:alt_block_scheme}). \\ 
			$\vvec$ & Vector formed by concatenating all subregions of $G$ \\
			$\vvec[i]$ & Vector formed by concatenating all regions of $\reg[i]$ \\ 
			$\vvec[i,j]$ & Vector on all vertices of $\subreg[i,j]$ \\
			$\wsize$ & Word size \\ \hline
		\end{tabular}
	\caption{Notations used in describing graph representation.}
	\label{tab:notation}
\end{table*}
\newpage

\begin{table*}[ht]
\centering
\begin{tabular}{l | l}
  Data Structure & Description \\ \hline
  $\fstofreg$ & A bit-vector of length $\dnumv$ the $k$'th bit of which \\
              & marks that $\vvec[k]$ as the first element of its region. \\
  $\fstofsubreg$ & A bit-vector of length $\dnumv$ the $k$'th bit of which \\
              & marks that $\vvec[k]$ as the first element of its sub-region. \\
  $\idxvec$ & Bit vector marking the number of sub-region boundary vertices per \\
              & region. \\
  $\bdvec$ & Bit vector of length $n'$ which marks if vertex $\vvec[k]$ is a \\
              & sub-region boundary vertex \\
  $\regvec$ & Vector each mapping sub-region boundary vertex to its defining \\
              & sub-region and label within that sub-region. The length of this\\
			  & vector is the sum of the sub-region boundary vertices for each \\
			  & region. \\
  $\subregvec$ & Vector recording the region label of all sub-region boundary \\
			  & vertices within the sub-regions. The length of this vector is \\
			  & the sum of the sub-region boundary vertices for each sub-region. \\
			  & (Note that due to duplicates $|\subregvec| > |\regvec|$.) \\ \hline
\end{tabular}
\caption{Data structures used to convert between graph, region, and sub-region labels.}
\label{tab:ds_label_conv}
\end{table*}


\begin{table*}[ht]
	\centering
		\begin{tabular}{ l | l}
			Data Structures & Description \\ \hline
			$\B_R$ & Bit vector of length $N$ marking region boundary vertices. \\
			$\B_S$ & Bit vector of length $N$ marking sub-region boundary vertices. \\
			$\S$ & Array storing all sub-regions, packed. \\
			$\F$ & Per sub-region bit-vector marking if sub-region is first sub-region in its region. \\
			$\D$ & Per sub-region bit-vector marking if sub-region starts in different block of $\S$ \\
			& than the preceeding sub-region. \\
			$\N_S$ & Array storing $\alpha$-neighbourhoods of all sub-region boundary vertices. \\
			$\N_R$ & Array storing $\alpha$-neighbourhoods of all region boundary vertices. \\
			$\L_S$ & Vector recording the minimum graph label of interior vertices of each sub-region. \\
			$\L_R$ & Vector recording the minimum graph label of interior vertices of each region. \\
			$N_{i,j}$ & The number of vertices store in sub-region $i,j$. \\
			 & (part of sub-region encoding) \\
			$\G_{i,j}$ & A succinct encoding of the graph structure of sub-region $R_{i,j}$. \\
			 & (part of sub-region encoding) \\
			$\B_{i,j}$ & A bit-vector marking vertices in a sub-region as either region or sub-region boundaries. \\
			 & (part of sub-region encoding) \\
			$\K_{i,j}$ & An array storing the $q$ bit-keys for each vertex within a sub-region. \\
			 & (part of sub-region encoding) \\
			$\G_x$ & A succinct encoding of the graph structure of $N_\alpha(x)$ (\ref{sec:datastructs}). \\
			 & (part of $\alpha$-neighbourhood encoding) \\
			$\T_x$ & A bit vector marking the terminal vertices in an  $\alpha$-neighbourhood. \\
			 & (part of $\alpha$-neighbourhood encoding) \\
			$p_x$ & Position of $x$ in $\pi_x$ (permutation of vertices in $N_\alpha(x)$\\
			 & (part of $\alpha$-neighbourhood encoding) \\
			$\K_x$ & An array storing keys associated with vertices in  $N_\alpha(x)$ . \\
			 & (part of $\alpha$-neighbourhood encoding) \\		
			$\L_x$ & An array stroring labels of the vertices in $N_\alpha(x)$. \\
			& (part of $\alpha$-neighbourhood encoding) \\ 
			$\regptable$ & $\alpha$-neighbourhood pointer table for \\
			& region boundary vertices (Section~\ref{sec:alt_block_scheme} ) \\
			$\subregptable$ & $\alpha$-neighbourhood pointer table for \\
			& sub-region boundary vertices (Section~\ref{sec:alt_block_scheme} ) \\ \hline
		\end{tabular}
	\caption{Data Structures used to represent graph components}
	\label{tab:data_structs}
\end{table*}

\newpage