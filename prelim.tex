% EDITING GUIDELINES
%
% * Limit lines to no more than 100 characters.  The default should be
%   80 characters.  This leaves room for the addition of 20 new
%   characters on the same line later on.
%
%   RATIONALE: Long lines are hard to read and cause trouble when
%   viewing two versions of the file side by side (e.g., using vimdiff)
%
% * Start every sentence on a new line.
%
%   RATIONALE: This makes sentences so much easier to find using forward
%   search in emacs or vim.
%
% * Do not justify paragraphs.  Particularly, when editing an existing
%   version of the text, do not change line breaks apart from adding
%   line breaks for lines that would otherwise be too long.
%
%   RATIONALE: This makes changes from previous versions so much easier
%   to identify, particularly when using a VCS.  Adding a word and
%   justifying the paragraph may change all the lines in the paragraph,
%   making this look like a huge edit.  Without justification, the
%   diff shows only the addition of the one word.

% ------------------------------------------------------------------------------
\section{Preliminaries}
% ------------------------------------------------------------------------------

% ------------------------------------------------------------------------------
\paragraph{Rank and select queries on bit vectors.}
% ------------------------------------------------------------------------------

As is the case for a wide range of succinct data structures, our data
structure relies on succinct representations of bit vectors
that allow constant-time rank and select queries.
Given a bit vector $S[1 \tdots N]$, and a bit value $x \in \{0, 1\}$, a
$\rankop[x]{S, i}$ operation returns the number of $x$s in $S[1 \tdots
i]$, while a $\selop[x]{S, r}$ operation returns the index of the
$r$th $x$ in~$S$.
In other words, $\selop[x]{S, r} = \min \set{i \mid \rankop[x]{S, i} = r}$.
The problem of representing a bit vector succinctly to support
$\rankopsym$ and $\selopsym$ operations in constant time in the word RAM model
with word size $\ThetaOf{\lg N}$ bits has been considered in
\cite{jac_1989,clark_96, DBLP:journals/talg/RamanRS07}, and these results can be
directly applied to the external memory model.
The following lemma summarizes the results of Jacobson~\cite{jac_1989} and
Clark and Munro~\cite{clark_96} (part (a)), and
Raman~\etal~\cite{DBLP:journals/talg/RamanRS07} (part (b)).

% ------------------------------------------------------------------------------
\begin{lemma}
  \label{lem:rank_select}
  A bit vector $S$ of length $N$ and containing $R$ $1$s can be
  represented using either (a) $N + \ohOf{N}$ bits or (b) $\lg \binom{N}{R}
  + \OhOf{N \lg \lg N / \lg N}$ bits to support the access to each bit, as
  well as $\rankopsym$ and $\selopsym$ operations, in $\OhOf{1}$ time (or
  $\OhOf{1}$ I/Os in external memory).\footnote{Note that $\log \binom{N}{R} +
    \OhOf{N \log \log N / \log N} = \ohOf{N}$ as long as $R = \ohOf{N}$.}
\end{lemma}
% ------------------------------------------------------------------------------

% ------------------------------------------------------------------------------
\paragraph{Planar graph partitions.}
% ------------------------------------------------------------------------------

Frederickson \cite{Frederickson87} introduced the notion of an
\emph{$r$-partition} of an $n$-vertex graph $G$, which is a collection of
$\ThetaOf{n / r}$ subgraphs of size at most $r$ such that for every
edge $xy$ of~$G$ there exists a subgraph that contains both $x$ and $y$.
A vertex $x$ is \emph{interior} to a subgraph if no other
subgraph in the partition contains $x$, otherwise $x$ is a
\emph{boundary vertex}.
Frederickson showed the following result for planar graphs.

% ------------------------------------------------------------------------------
\begin{lemma}[\cite{Frederickson87}]
  \label{lem:fred_graph_sep}
  Every $n$-vertex planar graph of bounded degree has an $r$-partition
  with $\OhOf{n / \sqrt{r}}$ boundary vertices.
\end{lemma}
% ------------------------------------------------------------------------------

%%% Local Variables:
%%% TeX-PDF-mode: t
%%% TeX-master: "main"
%%% End:
