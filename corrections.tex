

%--------------------------------------------------------------------------------
\section{Open Problems}\label{sec:open_problems}
%--------------------------------------------------------------------------------

This paper has presented data structures for representing planar graphs 
and triangulated terrains that are both succinct and efficient in the
external memory setting.
A possible extension of our technique for terrains would be to 
investigate if these techniques could be extended to tetrahedral meshes.
There are a number of succinct representations for triangulations which 
we have described in Section~\ref{sec:background}, including 
\cite{chuang_et_al_1998}, \cite{DBLP:conf/walcom/YamanakaN08}, 
\cite{DBLP:conf/isaac/BarbayAHM07}, \cite{DBLP:conf/wads/AleardiDS05}, 
and \cite{DBLP:journals/tcs/AleardiDS08}.
However, we are unaware of any similar representation for tetrahedral
meshes.
The approach we use relies on the planarity of the augmented dual graph
of a triangulation. 
This planarity condition will not hold for a tetrahedral mesh.

Efforts to represent triangulations succinctly fall into one of three
categories; $k$-page embeddings (based on Jacobson \cite{jac_1989}); 
canonical orderings of the vertices and edges (based 
on \cite{chuang_et_al_1998}); and triangulation
decomposition (based on \cite{DBLP:conf/wads/AleardiDS05}).
Encoding tetrahedral meshes presents challenges regardless of
which technique we select.

$k$-page embeddings rely on the fact that for planar graphs and
triangulations there is a page embedding for a small constant 
value of $k$.
It may be possible to show that a tetrahedral mesh can be 
embedded with a suitably small $k$, but this appears to be 
an open problem. 
Existing techniques for canonical orderings all rely on 
an embedding of the graph vertices in the plane, and thus 
this approach could be directly extended graphs in
higher dimensions.
Finally, the third approach decomposes the triangulation into suitably 
tiny triangulations
and represents these smallest traingulations based on a succinct
catalog of all possible tiny 
triangulations~\cite{DBLP:conf/wads/AleardiDS05}.
The additional complexity of a tetrahedral mesh, will create 
challenges in encoding such a catalog, and in combining the 
tiny tetrahedral meshes into a single structure.

 \section{Explanations for Corrections too Long for Margin Notes}

\begin{correction}\label{cor:j_index}
The original formula here was:

$\reglbl[i]{x} := \rankop[0]{\bdvec,k - \rankop[0]{\bdvec,\regpos{i}-1}} + \bdcount{i}$

Which I re-wrote as:

$\reglbl[i]{x} := \rankop[0]{\bdvec,k} - \rankop[0]{\bdvec,\regpos{i}-1} + \bdcount{i}$

I assume there was in error and that the $k$ is the last argument of the initial
rank operation.
\end{correction}




